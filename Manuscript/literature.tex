
In this section, we briefly review the literature relevant to our paper, connecting ideas from three domains: experimentation and its market-level implications, firm-level incentives to innovate, and the relationship between firm size and the diversity of approaches. These literatures provide a foundation for understanding how firms navigate their choice of experimental strategy and how their choices influence market-level outcomes. These literatures motivate our model in Section \ref{sec:model}, which explores how firm-level experimentation strategies shape market-level outcomes.

The first literature we draw on is the recent work on experimentation, which has thus far focused on the design, effectiveness, and decision-theoretic implications of experimental strategies. For example, \citet{koning2022experimentation} and \citet{gans2019foundations} emphasize how decision-makers design experiments to gather information that improves subsequent investments, product designs, or market strategies. In these studies, experimentation is valued primarily for reducing uncertainty and enabling better decision-making. However, our work shifts the focus to settings where experiments yield direct private payoffs but have broader market-level consequences. For instance, in our empirical context of clinical trials, additional successes do not increase the overall benefit once one firm succeeds, as the total value is shared among all successful firms. This market feature creates a gap between what benefits individual firms and what is best for society. Moreover, this gap depends on the composition of the experimenters. Thus, we extend the literature to explore market-level technological diversity and innovation.

Our paper also builds on the theoretical literature examining how firms choose experimental strategies. Our model shares similarities with \citet{nelson1961uncertainty}. However, while Nelson assumes a single attempt per approach, we do not impose such constraints, allowing for a more flexible representation of experimentation under uncertainty.  \citet{bryan2017direction} also explore firms’ choices among competing innovation approaches. Unlike them, we examine the case where some firms may follow multiple approaches but approaches could be dead ends. This extension captures the inherent uncertainty in experimental settings, where outcomes within a given approach correlate. We find that the market may have too little diversity, which aligns with \citet{dasgupta1987simple}, who demonstrate when firms ignore the effect of their investments on others, their research portfolios will be too similar to each other.

% \todo[inline]{there must be a ton of work on myopia and limited rationality and so on that would lead to herding (but is not directly about how herding differs between markets with single and multiple shot firms. We should briefly acknowledge that although we don't want those people as reviewers.}

The second domain we connect is the literature on firm-level incentives to innovate, particularly how private incentives can lead to herding and clustering in specific technological trajectories \citep{lieberman2006firms, krieger2021trials}. Herding has been extensively studied, notably in the work of \citet{bikhchandani1992theory}, which examines how agents may disregard private information to converge on specific actions based on observed behavior. In contrast, our framework focuses on static beliefs, where herding arises not from belief updating but from private incentives to pursue the most promising approaches. This behavior reflects static, payoff-driven incentives to converge on the most promising approach---distinct from informational cascades driven by belief updating. These dynamics highlight how private incentives can contribute to reduced diversity and suboptimal outcomes at the market level, underscoring the tension between individual and collective rationality in innovation.

Finally,  our work speaks to the relationship between firm size and the diversity of approaches. Larger firms, which can conduct multiple experiments simultaneously, face fundamentally different trade-offs than smaller firms constrained to one-off experiments. \citet{cohen1992tradeoff} argue that concentrating experiments within a few firms reduces the diversity of approaches. Our model diverges from this stream of research by considering non-additive payoffs, where the incremental value of success diminishes significantly with each additional success. This non-additivity affects diversity and innovation at the market level. %By incorporating these dynamics, we highlight the structural constraints and opportunities associated with firm size and their effects on the diversity of technological exploration.


% %% 1. Models of choice of approach
% This paper builds on the theoretical literature examining the firm choice of experimental strategy. Our model is closest to \citet{dasgupta1987simple}, who also consider the correlation in outcomes for experiments (or, in their case, research portfolios). They show the externalities private actors can exert on their competitors, leading to research portfolios that may be excessively correlated.  \citet{bryan2017direction} study the direction of innovation and model firm's choice among competing approaches. Differently from them, we allow for multiple approaches, each of which could be a dead-end. When the viability of an approach is uncertain, the outcomes of any experiment using that approach are correlated. Lastly, our model is similar in spirit to \citet{nelson1961uncertainty}, but Nelson implicitly limits the number of attempts per approach, whereas we do not impose this.


% %% 2. Herding
% Second, our theory and empirical results are related to the literature on herding \citet{bikhchandani1992theory}. Our theory differs in that we do not permit firms to update their beliefs. In our model, herding in the most promising approach arises because of private incentives rather than because of changes in beliefs. In our setting, experiments are directly about finding solutions rather than updating priors to guide subsequent investments.


% %% 3. Size and diversity 
% Third, our paper is connected to the literature on firm size and the diversity of approaches. \citet{cohen1992tradeoff} argue that greater experimenter scale reduces the diversity of approaches. Our results differ because we assume that the payoff is not additive, i.e., the incremental value of a second success is very small.

% %% 4. Experimentation

% Fourth, we contribute to the growing body of scholarship on experimentation. The literature has primarily focused on the effectiveness of experimentation \citep{koning2022experimentation}, and how a decision-maker should design their experiments \citep{camuffo2024scientific, gans2019foundations, gans2023experimental}. We focus on market-level outcomes rather than individual ones. In our setting, the incremental social value of a second success is zero. Thus, two successful firms would split the (fixed) social value. This creates a wedge between what is privately rational and what is collectively desirable. When firms carry out multiple experiments, this wedge becomes smaller.

% Our approach differs from much of the literature on experimentation in another respect as well. For us, a successful experiment directly yields a private payoff. This is a natural interpretation in our empirical context of clinical trials. In many other papers (e.g., \citet{gans2019foundations, camuffo2020scientific, koning2022experimentation} experiments create value by providing information that improves decisions, such as whether to invest or the choice of market or product design. Typically, these are decision-theoretic settings, where the payoff does not depend on the outcomes of experiments by others, unlike in our context.

% %% 5. Org literature 

% Our analysis also speaks to the organizational behavior literature on technological trajectories \citep{ciarli2021digital, nelson2023if, tan2023road}. We focus on incentives rather than firm capabilities, though we do control for capabilities in our empirical analysis.