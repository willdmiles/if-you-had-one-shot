%%%%%%%%%%%%%%%%%%%%%%%%%%% NULL MODEL %%%%%%%%%%%%%%%%%%%%%%%%%%%

% MOTIVATION
% PUZZLE
% SOLUTION
% FINDINGS
% IMPLICATIONS


From advancing healthcare to addressing climate change, solving society's most pressing challenges requires innovation and extensive experimentation. Breakthroughs often emerge when a wide range of different approaches are explored \citep{gross2023world}. In the context of technological progress, an approach can be understood as a structured hypothesis or method designed to explore specific pathways, mechanisms, or combinations of variables in solving a problem \citep{sorenson2024theory}. For instance, the development of statins was made possible by years of experimentation, including Akira Endo's groundbreaking work on fermented natural products to reduce cholesterol. Similarly, the invention of the transistor arose from systematic, extensive testing of material and design combinations across multiple labs, culminating in the 1947 breakthrough by John Bardeen, Walter Brattain, and William Shockley. Thomas Edison’s quest for the ideal filament for the light bulb involved testing thousands of combinations of materials and configurations. These examples illustrate how diverse experimentation increases the odds of solving complex problems.

Conventional wisdom holds that increasing the number of experiments leads to greater variety in the approaches used to solve complex problems \citep{prager2021seek, hong2004groups}. Underpinning this conventional wisdom is the belief that involving many small-scale experimenters is a more effective way to achieve approach diversity than relying on a few larger experimenters conducting multiple experiments each \citep{cohen1992tradeoff}. The logic is simple: more solvers are assumed to independently explore a wider range of approaches by bringing in diverse perspectives \citep{thomke1998modes}. However, understanding the incentives of firms is critical to determining whether they will explore new approaches or herd toward the most promising one. What should a firm do if it had \textit{only} one shot? All else equal, a firm in this position is likely to focus on the most promising approach, as this maximizes its perceived chance of success \citep{bikhchandani1992theory}.  In the aggregate, however, the concentration of efforts on the same approach diminishes diversity across experiments, ultimately reducing the overall likelihood of solving the problem. This raises the question: under what conditions can a greater diversity of approaches be achieved, and how does this influence the likelihood of success?

In this paper, we develop a simple model to examine how firms decide which approaches to pursue, distinguishing between multi-experiment firms and one-shot experimenters. In our model, experiments face two sources of failure: an approach may be a dead end, or its implementation may fail even if the approach is otherwise viable. Multi-experiment firms reduce the correlation between their outcomes by spreading experiments across separate approaches (with independent likelihoods of being viable), increasing the likelihood of at least one success. In contrast, one-shot experimenters tend to converge on the most promising approach, independently herding into the same strategy and reducing diversity at the market level. As a result, our model predicts that markets with a few multi-experiment firms will generate greater diversity in approaches than otherwise similar markets dominated by many one-shot experimenters. Diversity results in a higher probability of the market finding at least one successful solution, though the average experiment is less likely to succeed: Diversity reduces the probability of complete failure in return for a lower probability of success per experiment. This tradeoff between average and aggregate success is central to our argument and has important implications for both firm strategy and innovation policy.


%%%%%%%%%%%%%%%%%%%%%%%%%%% DATA AND KEY RESULT %%%%%%%%%%%%%%%%%%%%%%%%%%%

To test the implications of our model, we use a comprehensive dataset on early-stage pharmaceutical drug development from 1996 to 2023, which includes 49,866 drug development projects led by 3,845 firms across 241 therapeutic classes. These data capture the preclinical testing of drug-target combinations, where a therapeutic class-target pair represents an approach, and a firm’s trial of that pair constitutes an implementation attempt. This framework allows us to empirically examine two key hypotheses: (1) within a therapeutic class, as the average number of experiments conducted per firm increases, the diversity of approaches---measured as the variety of targets drugged---will rise; and (2) greater diversity of approaches will decrease the average success rate per trial while increasing the likelihood of at least one successful outcome within a therapeutic class.

Our empirical analysis supports these hypotheses. Markets with a one-unit increase in average experimenter scale exhibit a 3.4 standard deviation increase in target diversity. Additionally, a higher level of diversity is associated with a trade-off: a 20.3 percentage point increase in the likelihood of at least one experiment progressing to Phase 1 clinical trials, alongside a decline in individual experiment success rates. These results show that the experimenter scale can foster approach diversity and increase the probability of success in solving complex problems with uncertain solutions.

While the evidence supports our main theoretical predictions, other explanations could also account for the patterns we observe. For example, markets may differ in size, technological opportunity, or firm capabilities, which may affect the average scale of experiments and the likelihood of success. However, none of these can explain the core empirical associations we document, namely that the average experimenter scale is positively associated with diversity and positively associated with the probability of at least one successful solution but \textit{negatively} associated with average success per experiment. 

%We evaluate two competing explanations: (1) the possibility that increased diversity results from the discovery of new targets rather than the experimenter scale and (2) the role of firm characteristics, such as age, which may influence both the experimenter scale and approach diversity. For the first explanation, 

We control for the total number of targets, the number of firms, and the total number of projects in a therapeutic area (a higher level of aggregation than a market), along with therapeutic area and time fixed-effects. Our results are also similar when we additionally control for the rate of target discoveries within disease classes. While we find that single-experiment firms do indeed introduce a significant proportion of new targets, this does not diminish the relationship between experimenter scale and diversity. Furthermore, contrary to the assumption that one-off experiments are exclusive to startups, both large public firms and startups actively engage in single experiments.  Finally, to account for differences in firm capabilities and experience, we include the average age of firms in a therapeutic class as a control variable and find consistent results \citep{krieger2022missing}.

%Additionally, we employ an instrumental variables strategy to leverage quasi-exogenous variation in the experimental scale in a disease class to estimate our main effect. We use lagged values of the average experimenter scale as an instrument and find consistent results: a one-unit increase in the average experimenter scale leads to a significant rise in target diversity, as predicted by our model.

Taken together, these additional analyses support our main findings: experimenter scale plays a crucial role in enhancing approach diversity and reducing the likelihood of outright failure at the market level. Nevertheless, small-scale experimenters play a crucial role in introducing new targets to experimentation, with firms conducting only one experiment being more likely to explore novel targets than multi-experimenters. These complementary findings highlight a critical dynamic: while markets with greater average experimenter scale foster diversity by leveraging the broader experimentation efforts of large-scale firms, they also depend on the exploratory efforts of small-scale experimenters to expand the frontier of novel targets.

This paper makes three key contributions. First, to the literature on experimentation, we show that the scale of experimentation---how many shots each firm takes---fundamentally shapes market-level diversity and success. This shifts focus from the effectiveness of individual experiments to the collective architecture of experimentation. While the literature has primarily focused on the effectiveness of experimentation \citep{koning2022experimentation}, our paper shifts the focus to the scale of experimentation and explicitly relates it to market-level outcomes. We thus link experimentation to technology policy, particularly regarding the optimal societal organization of collective experimentation \citep{nelson1961uncertainty}.

Second, we offer a conceptual contribution by distinguishing between two components of an experiment: the approach and the implementation attempt. \citet{camuffo2024scientific} examine how scientific training can help entrepreneurs design better experiments. Implicitly, much of the literature on experimentation, including studies by \citet{koning2022experimentation} and \citet{gans2019foundations}, do not distinguish between the implementation and approach, instead bundling them into a single concept. We argue that these are distinct \citep{bryan2017direction, dasgupta1987simple}. An approach refers to a hypothesis or theory about the cause of a problem, suggesting potential ways to solve it \citep{sorenson2024theory}. In contrast, an implementation is a specific attempt to solve the problem based on the hypothesized cause. For success, both the approach must be viable and the implementation must be effective. As we demonstrate, the shared risk of failure across all implementation attempts using the same approach can lead to excessive herding.\footnote{Throughout the paper, we distinguish ``approach'' (a hypothesis about a solution pathway, such as targeting a specific protein) from ``implementation'' (a particular experimental effort based on that approach, such as testing a specific compound). This distinction is essential to our model, as it helps explain how correlated failures can arise even from independent trials.}

Third, our empirical results emphasize the distinct ways in which large- and small-scale experimenters affect diversity. Small-scale experimenters contribute by introducing novel approaches, thereby expanding the portfolio of available approaches. Large-scale experimenters, on the other hand, add diversity by more thoroughly investigating known but understudied approaches. This finding ties to the literature on technological trajectories, suggesting that markets with small-scale experimenters may underexplore some trajectories---herding towards the ones that are most likely to succeed \citep{ciarli2021digital, nelson2023if, tan2023road}.
