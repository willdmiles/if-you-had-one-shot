
This paper addresses the question of how market structure influences the diversity of experimental approaches firms take and, ultimately, the likelihood of solving innovation problems with uncertain solutions. To investigate this, we develop a theoretical model that builds on work by \citet{dasgupta1987simple} and \citet{bryan2017direction} that explores how firms choose between approaches under varying market structures, focusing on multi-experiment firms versus single-experiment firms. The model yields three key propositions at the market level. First, firms conducting multiple experiments are more likely to diversify their approaches, whereas many single-experiment firms tend to converge on the most promising approach, reducing market-level diversity. Second, markets dominated by single-experiment firms achieve higher success rates per experiment, due to their focus on the most promising approach. Third, multi-experiment firms maximize the probability of at least one success through diversification, implying that markets with greater experimenter scale are more likely to see at least one experimental success. To test these propositions, we use detailed data on early-stage pharmaceutical drug development, spanning 49,866 projects across 27 years, and demonstrate how experimenter scale drives approach diversity and influences market-level success.

The findings reveal a robust and significant relationship between experimenter scale and approach diversity. Markets with a higher average experimenter scale---where firms conduct multiple experiments---exhibit a nearly three standard deviation increase in target diversity for every one-unit increase in average scale. This effect is consistent across alternative econometric specifications. Importantly, markets with higher experimenter scale are more likely to achieve at least one successful outcome, with a one-unit increase in scale linked to a 25.9 percentage point rise in the likelihood of at least one experiment progressing to Phase 1 trials. However, this benefit comes at the cost of a lower average success rate per experiment, as larger-scale experimentation dilutes focus on the most promising approaches. Consequently, we also see that markets with greater approach diversity---driven by multi-experiment firms---are more likely to achieve at least one successful outcome, compared to markets dominated by single-experiment firms. These results are robust to alternative explanations, including controls for new target discovery and firm age, underscoring the role of multi-experiment firms in fostering a diverse innovation ecosystem while balancing trade-offs in efficiency and exploratory breadth. 

Our findings have implications for the broader literature on experimentation and innovation. By highlighting the dual roles of small- and large-scale experimenters, we add to the understanding of how diversity in approaches arises and evolves within markets \citep{cohen1992tradeoff}. Small-scale experimenters act as pioneers, introducing novel approaches that expand the frontier of possibilities, while large-scale experimenters provide depth by thoroughly testing and validating established approaches \citep{kotha2011entry, ahuja2001entrepreneurship}. This complementarity bridges gaps in the literature on technological trajectories, suggesting that innovation ecosystems benefit from balancing exploratory breadth with systematic refinement \citep{luger2018dynamic, fang2010balancing, chen2008rival}. Our distinction between approach and implementation further informs studies of experimentation by underscoring how shared risks in approach viability can amplify herding behaviors and potentially limit diversity in high-stakes contexts.

For policy, our results suggest a nuanced perspective on fostering innovation through experimentation. Encouraging both small- and large-scale experimenters within markets is crucial to achieving a balance between exploration and exploitation. Policies that support small-scale experimenters, such as grants for early-stage research or incubator programs, can help uncover novel targets and approaches \citep{bradley2021policy}. Meanwhile, mechanisms that enable larger firms to scale their experimentation, such as tax incentives or public-private partnerships, ensure that promising but understudied approaches receive the rigorous testing needed for broader application \citep{dimos2016effectiveness, lerner2009boulevard}. By designing interventions that maintain this balance, policymakers can enhance the robustness of innovation ecosystems and increase the likelihood of market-level breakthroughs.

These insights also have practical implications for the structure of research funding and the design of innovation ecosystems. Allocating resources to create spaces where small-scale experimenters can explore untested ideas while facilitating partnerships with larger organizations can improve innovation outcomes \citep{cappelen2012effects}. For instance, collaborative frameworks that combine the agility of startups with the resources of established firms may foster a more diverse and productive experimental landscape \citep{polidoro2021corporate, bhaskaran2009effort}. Similarly, strategies that mitigate herding, such as incentivizing exploration of high-risk, high-reward approaches, can prevent over-concentration on seemingly safe options, thus ensuring a more resilient innovation pipeline \citep{von2020exploration}. Together, these findings offer actionable pathways to strengthen the interplay between individual and collective experimentation in driving technological and societal progress.

While our findings provide valuable insights, several limitations warrant discussion. First, the causal interpretation of our results remains challenging. Although we use extensive controls, unobserved factors—such as market-specific shocks to funding, technology, or regulation—may bias our estimates. For example, simultaneous drivers of experimenter scale and diversity, like regulatory changes or scientific breakthroughs, complicate causal attribution. However, such factors are unlikely to account for the full set of empirical associations we document.

Second, our focus on the pharmaceutical industry, while offering rich data, limits generalizability. Pharmaceuticals involve high R\&D costs, long timelines, and significant uncertainty, which may amplify the dynamics we observe. Other industries, with different innovation cycles or competitive pressures, might display distinct patterns. Exploring these relationships in contexts like materials or renewable energy could test the broader applicability of our model and findings. However, we believe this limitation applies primarily to the empirical setting, not the underlying theory. With regard to our theory, our assumptions enforce two key scope conditions: (1) the total value from experimentation is fixed and independent from the \emph{number} of successes, and (2) that there is variation across firms in their ability to conduct multiple experiments. At first, these scope conditions may seem restrictive, but we believe that there are many markets and technologies---such as batteries, quantum computing, and medical devices---which look this way. In these industries, among others, experimentation is costly, which is one reason for variation in experimenter scale. Furthermore, all of these examples share the promise of temporary monopoly rents---that the first success will capture all value for a period of time.

Finally, deeper exploration of mechanisms and firm heterogeneity is needed. We highlight the roles of small- and large-scale experimenters but do not fully unpack how firm size, resources, or market structures influence their contributions to diversity and innovation. Future research should examine how these dynamics vary across industries and firm types, offering richer insights into the optimal design of innovation ecosystems.

By studying how market structure shapes collective experimentation outcomes, we hope this work encourages further analysis of how innovation systems can be designed to solve the complex and pressing problems.