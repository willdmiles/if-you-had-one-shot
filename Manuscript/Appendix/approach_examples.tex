\section{A Case Study on Approaches to Experimentation: Psoriasis}\label{app:case-study}

Psoriasis\footnote{https://www.niams.nih.gov/health-topics/psoriasis} is a chronic inflammatory skin condition affecting approximately 2-3\% of the global population \citep{Ayala-Fontanez}. A common symptom of the disease is the appearance of raised, silvery plaques \citep{Nestle-psoriasis}. Although treatments for psoriasis have been available for decades, the disease has many forms, and the causes of some less common types are still not fully understood \citep{guo2023signaling}. Many hypotheses exist, and some examples include: (i) overactive T-cells triggering inflammation and rapid skin cell production, (ii) genetic factors, such as mutations in the HLA-Cw6 gene, (iii) involvement of cytokines like IL-23 and IL-17, (iv) inflammatory lipid molecules called leukotrienes, and (v) abnormalities in keratinocytes that result in excessive skin cell production. This section describes examples of firms initiating pre-clinical trials for two distinct psoriasis drug development projects within the same year. These examples are sourced directly from Pharmaprojects and supplemented with details from Trialtrove.


\textbf{Astrazeneca.} In 2004, AstraZeneca began preclinical trials for two psoriasis treatments. One was a humanized antibody called Sifalimumab, which targeted interferon-alpha (IFN-$\alpha$). The underlying hypothesis was that in genetically predisposed individuals, the immune system is primed, and exogenous IFN-$\alpha$ may trigger psoriasis development.

At the same time, pre-clinical trials began for Certolizumab pegol, a recombinant humanized high-affinity anti-TNFalpha antibody fragment, developed by UCB (Celltech before the acquisition), for the treatment of chronic inflammatory conditions, including Crohn's disease (CD), rheumatoid arthritis (RA), psoriatic arthritis and ankylosing spondylitis 

The anti-TNF alpha psoriasis hypothesis suggests blocking Tumor Necrosis Factor-alpha (TNF-alpha). It has since been shown, however, that this approach can lead to the development or worsening of psoriasis, primarily due to an uncontrolled increase in type 1 interferons produced by plasmacytoid dendritic cells (pDCs), which are key players in psoriasis pathogenesis. 


\textbf{Stiefel Laboratories (GlaxoSmithKline).} In 2006, Stiefel Laboratories began pre-clinical trials for two psoriasis drugs. One was called Primolux, which was a 0.05\% topical formulation of the corticosteroid clobetasol, developed using its proprietary VersaFoam-EF technology, that targeted the nuclear receptor subfamily 3 group C member 1.

The second drug, Calcipotriol VersaFoam, was a vitamin D receptor antagonist. The treatment consisted of a 0.005\% topical formulation of calcipotriol, a vitamin D3 analog.

In 2009, GlaxoSmithKline acquired Stiefel Laboratories for \$2.9 billion to create a specialists dermatogolgy branch.\footnote{https://www.gsk.com/en-gb/media/press-releases/glaxosmithkline-completes-acquisition-of-stiefel} Consequently, in Pharmaprojects these projects are described as being originated by GlaxoSmithKline. This example highlights the complications in measurement. Pharmaprojects assigns the ``company developing a drug'' to each drug--treatment. Importantly, this would suggest that the listed focal company is both funding development and the primary decision maker.

Identifying the impact of such measurement error is hard to do. In the example of Stieffel laboratories, this measurement error would not affect ecometric estimates in our core results, at least. We would still consider the experiments to be done by a multi-experiment firm. If, however, we are on average more likely to incorrectly assign multi-experiment status to a large incumbent such as GlaxoSmithKline, then our analysis in Section \ref{subsec:which-firms} may inaccurately estimate the relationship between ownership type, the scale of experimentation, and the introduction of novelty (Figure \ref{fig:firm-heterogeneity} panels (a), (b), and (d)) 

However, measurement error in general would create an attenuation bias. In particular, if measurement error is also correlated with the target diversity dependent variable. If firm $X$ acquired two single experimenters, each with a distinct approach, and both experiments are attributed to firm $X$, then this would be a positive correlated measurement error leading to an upward bias on the estimated coefficient. Given our data, it is not feasible to scrutinize the history of each drug development project and the accuracy of originator firm assignment. This should be recognised as an empirical limitation of this study.